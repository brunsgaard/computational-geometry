\documentclass[10pt,a4paper,final,oneside,openany,article]{memoir}

% http://www.johndcook.com/blog/2009/09/14/latex-multi-part-definitions/
\newcommand{\twopartdef}[4] {
  \left\{
    \begin{array}{ll}
      #1 & \mbox{if } #2 \\
      {#3} & \mbox{if } #4
    \end{array}
  \right.
}
\input{prelude}

\title{Computational Geometry, Assignment 2}
\author{
    Jonas Brunsgaard \& Silas Ørting
}

\date{25th February 2015}

\bibliography{bibliography}

\begin{document}
\maketitle

\chapter*{Assignment 5.1}

\paragraph{Part 1}
We need to solve the recurrence
\begin{equation}
  \label{eq:Q}
  Q(n) = \twopartdef{O(1)}{n = 1,}{2 + 2Q(n/4)}{n > 1.}
\end{equation}
From case 1 of the Master theorem we have
\begin{align*}
  f(n) = O(n^c) \implies Q(n) = O(n^{\log_ba}),
\end{align*}
where $c < \log_ba$. If we look at the recursive part of (\ref{eq:Q}) we find
\begin{align*}
  a &= 2\\
  b &= 4\\
  \log_ba &= \log_42 = 1/2\\
  f(n) &= 2 = O(1) = O(n^0)
\end{align*}
so 
\begin{align*}
  Q(n) = O(n^{1/2}) = O(\sqrt{n})
\end{align*}

\paragraph{Part 2}
We need to show that $\Omega(\sqrt{n})$ is a lower bound for queries in the kd-tree. Consider the number of regions that are intersected by a horizontal query line. We assume that the line does not coincide with a horizontal split line. For simplicity we look at the case of a full kd-tree with the lowest layer being a horizontal split, that is $n = 2^k - 1$ where $k$ is an even positive integer, and the first split line is vertical.

\paragraph{}
A horizontal query line will intersect $O(\sqrt{n})$ regions. 

\paragraph{}
To see that this is the case, consider what happens when we add a new layer to the kd-tree. If the layer is vertical, then each region is split in two by a vertical line, so if the query line passed through a region it will now pass through two regions, and the number of intersections is doubled. If we the layer is horizontal, then each region is split in two by a horizontal line and since the query line does not coincide with a horizontal split line (by assumption) the number of intersections is the same. When we add a new layer $n$ is doubled, and the number of intersections is doubled when we have added two layers, so the number of intersections is $O(\sqrt{n})$.

\paragraph{}
Now we can construct a query rectangle by setting $x_1 = -\infty, x_2 = \infty$ and $y_1 < y_2$. The top and bottom of the query rectangle will intersect $O(\sqrt{n})$ regions, and each of the intersected regions will not be fully contained in the query rectangle so we will need to search the intersected region, giving us $\Omega(\sqrt{n})$ query time.



\chapter*{Assignment 5.10}

\paragraph{a} A solution to \emph{Range query counting} would be to keep
track of the number of leafs in any subtree. This can be done by adding a
\emph{counter} $v.c$ for each node $v$ (subtree) in the BST and when a new item is
inserted in the BST and "pass through the node on the way down the BST" these
counters are incremented, this additional step in construction will not impact
the construction time of $nlog(n)$. The idea is depicted in Figure 1. 
\begin{figure}[htbp]
    \centering
    \includegraphics[width=0.8\textwidth]{fig1.png}
    \caption{Datastructure for range counting queries.}
    \label{fig:1}
\end{figure}
Now, to perform a range query we simply make a small modification to the
\texttt{1DrangeQuery}, p. 97 in the book. We introduce a new variable $n$
variable and make the following change. 

\begin{itemize}
    \item On line 3, instead of reporing the point we increment $n$ by one if
$v_{\textrm{split}}$ is in the range.
    \item On line 8, instead of calling the subrouting \texttt{reportsubtree},
    we read and add the value of the counter of to $n$.
    $$n += lc(v).c$$
    \item Finally we return $n$.
\end{itemize}
As we get rid of the \texttt{ReportSubtree} procedure we eleminate the $k$
factor as we wanted. To prove that counting can be done in $log(n)$ time, we
observe that our data structure is a binary search tree. Thus a search can be
done in $log(n)$ time. To perform a range search we perform exactly two searches
to the bottom of the tree. Everything else other than the search down the tree
takes constant time and thus the finale result of is 
$$O(2log(n)) = O(log(n))$$

\paragraph{b} Using a range tree the idea from above can be generalized to
higher dimensions. The approach is that we for each node in the 1-level tree
$\mathcal{T}$, hold and associate data structure $\mathcal{T}_{assoc}$. The
associate data structure consists of the cannonical sets, as depicted in Figure 2.
\begin{figure}[htbp]
    \centering
    \includegraphics[width=0.8\textwidth]{fig2.png}
    \caption{Cannonical sets for $\mathcal{T}$ BST}
    \label{fig:2}
\end{figure}

These cannonical sets are used to build a new BST $\mathcal{T}_{assoc}$ sorted
on the $y$-coordinate, but otherwise, exactly as described in subassignment
\textbf{a}, for every node in $\mathcal{T}$.

When performing a range counting query we simply search $\mathcal{T}$ and
at every level we may search the 2-level data structure if the range of the
$x$-coordinates are within the search range. The search in an associate data
structure will return the count of points in the search range for that
cannonical set, and we simply add all these counts together on the way "down"
$\mathcal{T}$ to obtain the finale number of points in our search range.

A range counting query consists of two seaches (a constant we throw
away) in the $\mathcal{T}$. The length of the path in $\mathcal{T}$ is
at most $O(log(n)$. For each level we may perform a new binary binary
search in $\mathcal{T}_{assoc}$, and thus we obtain a time complexity of
$O(log(n)log(n)) = O(log^{2}n)$.

This can be generalized to higher dimensions, where each dimension will
require a factor of $O(log(n)$, thus we can conclude that range counting
queries in the $d$'th dimension can be performed in $O(log^{d}n)$ time.

\end{document}

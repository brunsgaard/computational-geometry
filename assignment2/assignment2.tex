\documentclass[10pt,a4paper,final,oneside,openany,article]{memoir}

% http://www.johndcook.com/blog/2009/09/14/latex-multi-part-definitions/
\newcommand{\twopartdef}[4] {
  \left\{
    \begin{array}{ll}
      #1 & \mbox{if } #2 \\
      {#3} & \mbox{if } #4
    \end{array}
  \right.
}
\input{prelude}

\title{Computational Geometry, Assignment 1}
\author{
    Jonas Brunsgaard \& Silas Ørting
}

\date{18th February 2015}

\bibliography{bibliography}

\begin{document}
\maketitle

\chapter*{Assignment 5.1}

\paragraph{Part 1}
We need to solve the recurrence
\begin{equation}
  \label{eq:Q}
  Q(n) = \twopartdef{O(1)}{n = 1,}{2 + 2Q(n/4)}{n > 1.}
\end{equation}
From case 1 of the Master theorem we have
\begin{align*}
  f(n) = O(n^c) \implies Q(n) = O(n^{\log_ba}),
\end{align*}
where $c < \log_ba$. If we look at the recursive part of (\ref{eq:Q}) we find
\begin{align*}
  a &= 2\\
  b &= 4\\
  \log_ba &= \log_42 = 1/2\\
  f(n) &= 2 = O(1) = O(n^0)
\end{align*}
so 
\begin{align*}
  Q(n) = O(n^{1/2}) = O(\sqrt{n})
\end{align*}

\paragraph{Part 2}
We need to show that $\Omega(\sqrt{n})$ is a lower bound for queries in the kd-tree. Consider the number of regions that are intersected by a horizontal query line. We assume that the line does not coincide with a horizontal split line. For simplicity we look at the case of a full kd-tree with the lowest layer being a horizontal split, that is $n = 2^k - 1$ where $k$ is an even positive integer, and the first split line is vertical.

\paragraph{}
A horizontal query line will intersect $O(\sqrt{n})$ regions. 

\paragraph{}
To see that this is the case, consider what happens when we add a new layer to the kd-tree. If the layer is vertical, then each region is split in two by a vertical line, so if the query line passed through a region it will now pass through two regions, and the number of intersections is doubled. If we the layer is horizontal, then each region is split in two by a horizontal line and since the query line does not coincide with a horizontal split line (by assumption) the number of intersections is the same. When we add a new layer $n$ is doubled, and the number of intersections is doubled when we have added two layers, so the number of intersections is $O(\sqrt{n})$.

\paragraph{}
Now we can construct a query rectangle by setting $x_1 = -\infty, x_2 = \infty$ and $y_1 < y_2$. The top and bottom of the query rectangle will intersect $O(\sqrt{n})$ regions, and each of the intersected regions will not be fully contained in the query rectangle so we will need to search the intersected region, giving us $\Omega(\sqrt{n})$ query time.



\chapter*{Assignment 5.10}

\end{document}

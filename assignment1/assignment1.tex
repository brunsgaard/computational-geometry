\documentclass[10pt,a4paper,final,oneside,openany,article]{memoir}


%BASIC PACKAGES
\usepackage[english]{babel} % last language decides document language!
\usepackage[utf8]{inputenc}          %text encoding
\usepackage[T1]{fontenc}
\usepackage[british]{isodate}

% Layout Fixes
\usepackage{booktabs} % nicer spacing between table rulers
\usepackage{microtype}
\usepackage{fixltx2e} % To prevent the figures from being placed
                      % out-of-order with respect to their
                      % "non-starred" counterparts


\usepackage{amsmath,amssymb, amsbsy}
\usepackage[amsmath,amsthm,thmmarks]{ntheorem}
\usepackage{semantic, stmaryrd}
\usepackage{graphicx}
\usepackage{todonotes}
\presetkeys{todonotes}{inline}{}
\usepackage{enumitem}


% Bibliography
%\usepackage[style=alphabetic,natbib=true]{biblatex}
\usepackage[backend=biber,style=numeric,natbib=true, defernumbers=true]{biblatex}

% Fonts
\usepackage{palatino}
\linespread{1.05}
\usepackage[scaled]{beramono}

% customize chapter pages
\makepagestyle{myheadings}
\makepagestyle{myheadingschapterpage}
\makeevenfoot{myheadingschapterpage}{}{\thepage}{}
\makeoddfoot{myheadingschapterpage}{}{\thepage}{}
\aliaspagestyle{chapter}{myheadingschapterpage}
\aliaspagestyle{title}{myheadingschapterpage}
\makeevenhead{myheadings}{}{%\scshape \thetitle
}{}
\makeoddhead{myheadings}{}{
%\footnotesize\scshape \thetitle%
}{}
\makeevenfoot{myheadings}{}{\thepage}{}
\makeoddfoot{myheadings}{}{\thepage}{}
\pagestyle{myheadings}

\def\thefigure{\arabic{figure}}
\setcounter{tocdepth}{0}


\setcounter{secnumdepth}{1}
\setcounter{chapter}{0}
\setsecheadstyle{\large\bfseries\raggedright}
\setsubsecheadstyle{\bfseries}



% Figures
\graphicspath{{./figures/}}
\usepackage{float}
\newsubfloat{figure}

\usepackage{sidecap}
\usepackage{caption}
\captionsetup{margin=0pt, font=small, labelfont=bf, format=hang}
% \setlength{\abovecaptionskip}{0pt}
% \setlength{\belowcaptionskip}{0pt}


% Default commands
\newcommand{\subimgwidth}{.48\textwidth}
\newcommand{\imgwidth}{.85\textwidth}

%
\theoremstyle{plain}
\theoremsymbol{\tiny $\Box$}
\newtheorem{Definition}[equation]{Definition}


%%% Local Variables:
%%% mode: latex
%%% TeX-master: "master"
%%% End:


\title{Computational Geometry, Assignment 1}
\author{
    Jonas Brunsgaard \& Silas Ørting
}

\date{18th February 2015}

\bibliography{bibliography}

\begin{document}
\maketitle

\chapter*{Assignment 1.3}
Brilliant stuff goes here...

\chapter*{Assignment 1.6}
Brilliant stuff goes here...

\chapter*{Chan}
Brilliant stuff goes here...

\chapter*{3.8}
When we insert a new edge in the DCEL we split a face into two faces, this means we have to update the incident-face information in $O(n)$ half-edge records. In the case of the polygon-partitioning algorithm, we do not need to store the face information, so adding a half-edge only requires that we make local updates to the four half-edges of the face we are splitting that has origin and destination at the vertices that are connected by the new edge. 

\chapter*{3.14 Visibility polygon}
Let $P$ be the polygon that is visible from $q$. If we traverse the edges of $P$ in CCW order then it holds that the angle between the lines $qp_i$ and $qp_{i+1}$ lie in $[0, \pi]$ and that angular sum is $2\pi$.

If we can find a point $p_0$ that is visible from $q$ then we can find the visibility polygon by starting in $p_0$ and adding vertices that satisfy the local angle constraint.

When a vertex $p_i$ that does not satisfy the local constraint is encountered we need to consider two cases. 
\begin{enumerate}
\item We have just turned right, then $p_i$ is obscured and we skip it.
\item We have just turned left, then the line to $p_i$ obscures some of the previous vertices, and we need to backtrack and discard previously accepted vertices untill the constraint is satisfied or we make a right turn. 
\end{enumerate}

The global angle constraint is eventually satisfied because we start from an initially visible vertex. If we have have violated the global angle constraint, then at some point we need to violate the local constraint in order to end up at the initally visible vertex.

\paragraph{}
Let {\verb visible } be a stack containing the vertices believed to be visible. Initially this contains the known visible point. Pseudocode is given below
\begin{verbatim}
p_1 = p_0
for i = 1 to N
  p = p_1.next
  skip = false
  while angle(q,p_1,q,p) > pi
    if is_left_turn(p_1.prev, p_1, p)
      p_1 = p_1.prev
      p.prev = p_1
    else
      p = p.next
      p_1.next = p
      p.prev = p_1
      skip = true
    end
  end
  if not skip
    p_1 = p
  end
end
\end{verbatim}
This runs in linear time, since each vertex is added at most once and removed at most once, and the angle constraint and turn direction can be computed in constant time.

\paragraph{}
Finding the visible starting point can be done in $O(n \log n)$ by triangulating the polygon and locating the triangle that $q$ falls in. We can then pick any of the three vertices as $p_0$.

\end{document}

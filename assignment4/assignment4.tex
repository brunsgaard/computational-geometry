\documentclass[10pt,a4paper,final,oneside,openany,article,oldfontcommands]{memoir}


%BASIC PACKAGES
\usepackage[english]{babel} % last language decides document language!
\usepackage[utf8]{inputenc}          %text encoding
\usepackage[T1]{fontenc}
\usepackage[british]{isodate}

% Layout Fixes
\usepackage{booktabs} % nicer spacing between table rulers
\usepackage{microtype}
\usepackage{fixltx2e} % To prevent the figures from being placed
                      % out-of-order with respect to their
                      % "non-starred" counterparts


\usepackage{amsmath,amssymb, amsbsy}
\usepackage[amsmath,amsthm,thmmarks]{ntheorem}
\usepackage{semantic, stmaryrd}
\usepackage{graphicx}
\usepackage{todonotes}
\presetkeys{todonotes}{inline}{}
\usepackage{enumitem}


% Bibliography
%\usepackage[style=alphabetic,natbib=true]{biblatex}
\usepackage[backend=biber,style=numeric,natbib=true, defernumbers=true]{biblatex}

% Fonts
\usepackage{palatino}
\linespread{1.05}
\usepackage[scaled]{beramono}

% customize chapter pages
\makepagestyle{myheadings}
\makepagestyle{myheadingschapterpage}
\makeevenfoot{myheadingschapterpage}{}{\thepage}{}
\makeoddfoot{myheadingschapterpage}{}{\thepage}{}
\aliaspagestyle{chapter}{myheadingschapterpage}
\aliaspagestyle{title}{myheadingschapterpage}
\makeevenhead{myheadings}{}{%\scshape \thetitle
}{}
\makeoddhead{myheadings}{}{
%\footnotesize\scshape \thetitle%
}{}
\makeevenfoot{myheadings}{}{\thepage}{}
\makeoddfoot{myheadings}{}{\thepage}{}
\pagestyle{myheadings}

\def\thefigure{\arabic{figure}}
\setcounter{tocdepth}{0}


\setcounter{secnumdepth}{1}
\setcounter{chapter}{0}
\setsecheadstyle{\large\bfseries\raggedright}
\setsubsecheadstyle{\bfseries}



% Figures
\graphicspath{{./figures/}}
\usepackage{float}
\newsubfloat{figure}

\usepackage{sidecap}
\usepackage{caption}
\captionsetup{margin=0pt, font=small, labelfont=bf, format=hang}
% \setlength{\abovecaptionskip}{0pt}
% \setlength{\belowcaptionskip}{0pt}


% Default commands
\newcommand{\subimgwidth}{.48\textwidth}
\newcommand{\imgwidth}{.85\textwidth}

%
\theoremstyle{plain}
\theoremsymbol{\tiny $\Box$}
\newtheorem{Definition}[equation]{Definition}


%%% Local Variables:
%%% mode: latex
%%% TeX-master: "master"
%%% End:


% http://www.johndcook.com/blog/2009/09/14/latex-multi-part-definitions/
\newcommand{\twopartdef}[4] {
  \left\{
    \begin{array}{ll}
      #1 & \mbox{if } #2 \\
      {#3} & \mbox{if } #4
    \end{array}
  \right.
}

\newcommand{\floor}[1] {\lfloor #1 \rfloor}
\newcommand{\D}{\mathcal{D}}


\title{Computational Geometry, Assignment 4}
\author{
    Jonas Brunsgaard \& Silas Ørting
}
\begin{document}
\maketitle


\chapter*{Assignment 7.1}
Let $X$ be a set of points in the plane that form a regular convex polygon,
and construct a point $q$ such that it is the midpoint of the regular convex
polygon, we now contruct a new point set $P$, $$ P=X \cup \{q\} $$ and denote
the number of points in $P$ by $n$.
Then the site $S_q$ in the Vor$(P)$ will have $n-1$ vertices.

Now draw a line segment from $q$ to some $p_i$ and a line segment from $S$ to
some $p_j$, remember that $P\{q\}$ is a regular convex polygon and $q$ is the
midpoint, thus $|qp_i$ = $|qp_j$. Choose $M$ to be the midpoint of $|qp_i$.
Due to the triangle inequality $|qM| = |Mp_i| < |Mp_j|$. Now, there exists
largest empty a circle as described in Theorem 7.4 which must contain $q$ and
$p_i$ on its boundaries. Because $|qM|=|Mp_1|$ they will share an edge. This
is true for all the points in $P\{q\}$. Thus $S_q$ must have $n-1$ edges and
because $S_q$ is bounded (this is trivial due to the regular convex polygon
of$P\{q\}$) $S_q$ it will contain $n-1$ vertices.


\chapter*{Assignment 7.3}
Let $X \subset \mathbb{R}$ and define the transformation $T : (x) \to (x,
x^2)$ that maps the real line to a parabola. Let $Y = \{T(x) | x \in R\}$.
Because $T$ is convex all points in $Y$ will lie on the convex curve and thus
the convex hull. We can report the points in sorted order by starting at the
leftmost point in the convex hull and moving CCW around the convex hull.
Locating the leftmost point and walking the convex hull takes $O(n)$ time.
We can recover $X$ in sorted order by applying $T^{-1}$ to the sorted points
in $O(n)$ time. As shown in assignment 7.12 we can find the convex hull from
the Voronoi diagram in $O(n)$ time. Thus, reduction can be used to reduce the
computation of Voronoi diagrams to the convex hull contruction, and further to
sorting. This implies that $\Omega(n \log n)$ is a lower bound for computing
the Voronoi diagram.

\chapter*{Assignment 7.10}
Compute $Vor(P)$ in $O(n \log n)$ time. From theorem 7.4 (ii) we have that
the two closest points must share an edge in $Vor(P)$. The number of edges in
$Vor(p)$ is at most $3n-6$ (Theorem 7.3) and thus $O(n)$. If we store
$Vor(P)$ in a DCEL and store the coordinates of the original points in the
face records, then we can calculate the distance between all neighbouring
points, and thus the smallest distance, in $O(n)$. Total running time is $O(n
\log n)$ from computing $Vor(P)$.


\chapter*{Assignment 7.12}
The convex hull of $P$ can be found by connecting those Vorononi cells that
are unbounded, or in the case where we have a bounding box, those cells that
share an edge with the bounding box. We can walk around the the edges of the
bounding box, by following the half-edges of the unbounded face. Assume we are
given a pointer to a half-edge on the boundary that has the unbounded face
as its incident face. We can detect if a face is unbounded by checking if
the outer component of the face is null. The number of edges on the boundary
is $O(n)$. So we can walk through the half-edges on the boundary and get the
convex hull of $P$ in $O(n)$ time.


\chapter*{Assignment 9.11}
We denote the Delaunay triangulation by $\D$.
\paragraph{a)}
Assume that the EMST contains an edge $e \notin D$. Let $p_i$ and $p_j$ be the
points connected by $e$ and consider the circle with $e$ as diameter. From
theorem 9.6 (ii) there must be a point, $p_k$, inside the circle. This point
must be connected to $p_i$ and $p_j$. If we consider the cycle defined by $p_i
\to p_j \to p_k \to p_i$, then we see that $e$ is the longest edge of this
cycle, and by the cycle property of MSTs $e$ cannot be in the EMST. This is a
contradiction so there cannot be an edge in EMST that is not in $\D$.


\paragraph{b)}
$\D$ is a planar graph which satisfies
\begin{align*}
  n \ge 3 \implies n_e \le 3n - 6,
\end{align*}
where $n$ is the number of vertices and $n_e$ the number of edges. We can use
Kruskals algorithm to find an MST in $O(n_e \log n_e)$. For a planar graph
$O(n_e \log n_e) = O(n \log n)$. We can compute $\D$ in $O(n \log n)$ expected
time, so we can find EMST in $O(n \log n)$ expected time.

\printbibliography
\end{document}

\documentclass[10pt,a4paper,final,oneside,openany,article,oldfontcommands]{memoir}

\input{prelude}

% http://www.johndcook.com/blog/2009/09/14/latex-multi-part-definitions/
\newcommand{\twopartdef}[4] {
  \left\{
    \begin{array}{ll}
      #1 & \mbox{if } #2 \\
      {#3} & \mbox{if } #4
    \end{array}
  \right.
}

\newcommand{\floor}[1] {\lfloor #1 \rfloor}
\newcommand{\D}{\mathcal{D}}


\title{Computational Geometry, Assignment 4}
\author{
    Jonas Brunsgaard \& Silas Ørting
}
\begin{document}
\maketitle


\chapter*{Assignment 7.1}
Assume $n > 3$. There will be one cell in $Vor(P)$ with $n-1$ vertices if the points such are distributed such that there is a point $q_0$ and all other points are distributed evenly on the boundary of $C_P(q_0)$ (the largest empty circle centered at $q_0$).

From Theorem 7.4 a we have
\begin{align*}
  q \in Vor(P) \iff C_P(q) \text{ contains three or more vertices on its boundary}
\end{align*}
Let $q_i$ and $q_j$ be two neighbouring points on the boundary of $C_P(q_0)$ ($\partial C_P(q_0)$) and let $q_{ij} \in \partial C_p(q_0)$ be the midpoint between $q_i$ and $q_j$. Let $n$ be the ray from $q_0$ through $q_{ij}$. Let $C_0$ be the circle defined by the diameter $q_0q_{ij}$, this circle intersects $\partial C_P(q_0)$ in $q_{ij}$. Now extend $C_0$ by moving $q_{ij}$ away from $q_0$ along $n$. Eventually $\partial C_0$ will intersect $\partial C_P(q_0)$ at $q_i$ and $q_j$. From theorem 7.4 we have that the center of $C_0$ will be a vertex in Vor(P). We can do this for each pair of neighbouring points and there are $n-1$ such pairs, so we get $n-1$ vertices in the cell for $q_0$.

\chapter*{Assignment 7.3}


\chapter*{Assignment 7.10}


\chapter*{Assignment 7.12}


\chapter*{Assignment 9.5}


\chapter*{Assignment 9.11}
We denote the Delaunay triangulation by $\D$.
\paragraph{a)}
Assume that the EMST contains an edge $e \notin D$. Let $p_i$ and $p_j$ be the points connected by $e$ and consider the circle with $e$ as diameter. From theorem 9.6 (ii) there must be a point, $p_k$, inside the circle. This point must be connected to $p_i$ and $p_j$. If we consider the cycle defined by $p_i \to p_j \to p_k \to p_i$, then we see that $e$ is the longest edge of this cycle, and by the cycle property of MSTs $e$ cannot be in the EMST. This is a contradiction so there cannot be an edge in EMST that is not in $\D$.


\paragraph{b)}
$\D$ is a planar graph which satisfies
\begin{align*}
  n \ge 3 \implies n_e \le 3n - 6,
\end{align*}
where $n$ is the number of vertices and $n_e$ the number of edges. We can use Kruskals algorithm to find an MST in $O(n_e \log n_e)$. For a planar graph $O(n_e \log n_e) = O(n \log n)$. We can compute $\D$ in $O(n \log n)$ expected time, so we can find EMST in $O(n \log n)$ expected time.

\printbibliography
\end{document}

\documentclass[10pt,a4paper,final,oneside,openany,article,oldfontcommands]{memoir}

\input{prelude}

% http://www.johndcook.com/blog/2009/09/14/latex-multi-part-definitions/
\newcommand{\twopartdef}[4] {
  \left\{
    \begin{array}{ll}
      #1 & \mbox{if } #2 \\
      {#3} & \mbox{if } #4
    \end{array}
  \right.
}

\newcommand{\floor}[1] {\lfloor #1 \rfloor}

\newtheorem{pinQ}{Theorem}
\newtheorem{pinQtime}[pinQ]{Theorem}

\newtheorem{rayshooting}{Theorem}

\addbibresource{bibliography.bib}

\title{Computational Geometry, Assignment 3}
\author{
    Jonas Brunsgaard \& Silas Ørting
}
\begin{document}
\maketitle

\chapter*{Assignment 6.5}
We are given a convex polygon $\mathcal{P}$ as an array of its $n$ vertices in
CCW sorted order [$p_0, p_1,p_2, \dots, p_n$].

We now imagne a line segment between $p_0$ and $p_{\lceil n/2 \rceil}$. This
will split our convex polygon into two smaller convex polygons. If $q$ is
below the line $(p_0, p_{\lceil n/2 \rceil}$ we continue our search in the
convex polygon below, otherwise we continue our search in the convex polygon
above. We continue this aproach until we after at most $log(n)$, subdivisions
get a triangle. Because $\mathcal{P}$ is convex we know that every combination
of vertices lies inside $\mathcal{P}$.

At this point we can check in $O(3)$ if the point $q$ is in the triangle. if
that is the case we know that $q$ is inside our polygon $\mathcal{P}$.


\chapter*{Assignment 6.6}
We need to test if a query point $q$ is inside a y-monotone polygon $P$ in
$O(\log n)$ time. If $q$ is below the top vertex in $P$ and above the bottom
vertex in $P$ then for each side of $P$ there will be a pair of adjacent
vertices bounding the $y$-coordinate of $q$. Let the line segment defined by
the left pair be denoted $e_l$ and the other $e_r$. Let $e_a$ be a horisontal
line going through the vertex above $q$ with the lowest $y$-coordinate, and
let $e_b$ be a horisontal line through the vertex below $q$ with the greatest
$y$-coordinate.

\begin{pinQ}
  Iff $q$ is in $P$ it will lie in the trapezoid defined by the intersection of $e_l, e_r, e_a, e_b$.  
\end{pinQ}

\begin{proof}
  Assume $q$ is in $P$, then by construction it will be below $e_a$ and above $e_b$. Since $q$ is in $P$ it must lie to the right of the left side of $P$ and to the left of the right side of $P$. Since the left and right side of the trapezoid coincides with the left and right side of $P$, $q$ is bounded by the four sides of the trapezoid and so must lie in it.
  
  Assume $q$ lies in the trapezoid. By construction the trapezoid is fully contained in $P$ so $q$ also lies in $P$.
\end{proof}

We can find the vertices defining $e_l$ and $e_r$ by binary search in the array.
We can determine if a vertex is on the left side, the right side, the top or the bottom by looking at the y-coordinate of the previous and next vertex. 
Assume the first entry in the array is the top vertex of $P$. We can find the bottom vertex by binary search, going right when we are on the left side and left when we are on the right side. If $q$ is above the top vertex or below the bottom vertex it is not in $P$ and we are done. Otherwise we need to find $e_l$ and $e_r$. This can be done by binary search on each of the sides separately. 

\begin{pinQtime}
  Determining if $q$ is in $P$ can be done in $O(\log n)$ time
\end{pinQtime}

\begin{proof}
  We need one binary search to find the bottom vertex taking $O(\log n)$ time. We need two binary searches to find $e_l$ and $e_r$ taking $O(\log n)$ time. Determining if $q$ is in the trapezoid can be done in $O(1)$ time. So the total time is $O(\log n)$.
\end{proof}

\chapter*{Assignment 6.7}
We are given a starshaped polygon $\mathcal{P}$ as an array of its $n$ vertices in
CCW sorted order [$p_0, p_1,p_2, \dots, p_n$]. and a point $p_\textrm{star}$.

The idea here is the same as in 6.5, we will divide the star-shaped polygon
into smaller star-shaped polygons, until we get a triangle, and when we have
the triangle we will test if $q$ is in the triangle.

We now imagne a line segment between $p_\textrm{star}$ and $p_0$ and another
line segment between $p_\textrm{star}$ and $p_{\lceil n/2 \rceil}$.

Now we determine which one of the two new star-shaped polygon, $q$
\emph{may} be part of. This can be done using vector normalization and
dot products to determine angles between $\vec{p_\textrm{star}p_1}$,
$\vec{p_\textrm{star}p_{\lceil n/2 \rceil}}$ and $\vec{p_\textrm{star}q}$.§

We continue this aproach until we after at most $log(n)$, divisions of
$\mathcal{P}$ get a polygon which is a triangle. We can now test in constant
time if $q$ lies inside this triangle and thus $\mathcal{P}$.

\chapter*{Assignment 6.16}
We need to define a data structure that can be used for the vertical ray shooting problem in 2D. For a set of line segments we look at the projection onto the x-axis. The projection should be such that if a line segment $l_i$ is below another segment $l_j$ then $l_i$ shadows $l_j$ partly or fully. This wil give a division of the x-axis which reflects which line segment is first intersected by a vertical ray coming from below. We can store this division in a sorted array and locate the intersected segment by binary search. Each entry in the array is a structure that stores the endpoints and the label of the projected segment.

\paragraph{}
We consider line segments to be closed, which means we have four different types of projected segments: open, closed, half-open to the left and half-open to the right. We need to store this information in the array. We consider lines that meet to be intersecting. Treating segments as closed allows us to treat vertical segments in the same way as non-vertical segments.

\paragraph{Storage}
\begin{rayshooting}
  Storing the projected line segments requires $O(m)$ space where $m \le 2n-1$ is the number of projected line segments.
\end{rayshooting}
\begin{proof}
We can show this by induction. We look at the segments ordered by decreasing y-coordinate and project them onto the x-axis one by one. For $n = 1$ we have $m = 1 \le 2n -1$. For the inductive case assume we have $n-1$ segments and $m \le 2(n-1) - 1 = 2n - 3$. When we add the $n$'th segment ($l^n$) it can fully shadow segments, partially shadow segments or not shadow segments.

If $l^n$ does not shadow any segments it gives rise to a single projected segment, so $m \le 2n - 3 + 1 \le 2n - 1$.

If $l^n$ fully shadows some other segment, the number of projected segments will be reduced.

For the last case we observe that at most two segments can be partially shadowed. There can be $O(n)$ fully shadowed segments between be the partially shadowed segments, but since these decrease the number of projected segments we consider the case where no segments are fully shadowed. We have two cases, one segment is partially shadowed or two segments are partially shadowed. 

\begin{description}
\item[One segment shadowed] Let $l^s$ be the partially shadowed segment. If both end points of $l^n$ lies inside $l^s$ we get three new segments $(l^s_{left},l^n_{left})$, $(l^n_{left}, l^n_{right})$, $(l^n_{right}, l^s_{right})$ and we remove the segment $(l^s_{left}, l^s_{right})$, so we add two segments. If only a single endpoint of $l^n$ lies inside $l^s$ we add two new segments and remove one.
\item[Two segments shadowed] Let $l^l$ be the left shadowed segment and $l^r$ the right shadowed segment. When we add $l^n$ we get the new segments $(l^l_{left},l^n_{left})$, $(l^n_{left}, l^n_{right})$, $(l^n_{right}, l^r_{right})$ and remove the segment $(l^l_{left}, l^r_{\right})$ resulting in two new segments.
\end{description}
So we add at most two new segments, so $m \le 2n - 3 + 2 = 2n -1$.
\end{proof}


\paragraph{Query time}
\begin{rayshooting}
  We can locate the first segment that is intersected by a vertical ray in $O(\log n)$ time.
\end{rayshooting}
\begin{proof}
We use binary search to locate the segment closest to the ray. We can check if a segment is intersected by the ray in constant time resulting in a $O(\log n)$ query time.
%  We have the projected segments stored in an array sorted by increasing $x$-coordinate. We can find the intersected line segment using binary search which takes $O(\log n)$.
\end{proof}


\paragraph{Preprocessing}
We can build the array using plane sweep along the $x$-axis. Assume line segments are defined such that the vertex with lowest $x$-coordinate is given first. If this is not the case, then we can redefine the segments in $O(n)$ time. First we store the vertices in a priority queue ordered by $x$-coordinate.

We must keep track of the segments intersected by the sweep line. We need to know which segment is currently visible, so we store the intersected segments in a balanced binary search tree ordered by $y$-coordinate.
When a start vertex is encountered the segment is added to the tree, if the newly added segment is below the currently visible segment an entry is added to the array containing the segment that was previously visible.
When an end vertex is encountered the corresponding segment is removed from the tree, if this segment was the visible segment an entry is added to the array.

\begin{rayshooting}
  The vertical ray shooting array can be build in $O(n \log n)$ time.
\end{rayshooting}
\begin{proof}
  Building the priority queue takes $O( n \log n)$ time. 
  When a start vertex is encountered a segment is added to the search tree, this takes $O(\log n)$. When an end vertex is encountered a segment is removed from the search tree, this takes $O(\log n)$. So for each segment we need $O(\log n)$ giving a total preprocessing time of $O(n \log n)$.
\end{proof}

\subsection*{Intersecting line segments}
We can reuse the data structure and query algorithm, but we need to change the construction algorithm. TODO: Explain new construction algorithm.




\chapter*{Assignment 10.1}
The primary structure is a balanced binary search tree ordered by y-coordinate. Each inner node $v$ stores an interval tree that indexes the canonical subset $P(v)$ on x-coordinate.

\subsection*{a. Algorithm}

\paragraph{Construction}
We can adapt the \textsc{Build2DRangeTree} algorithm to construct a range tree with associated interval trees. This gives us a recursive algorithm that takes a set of intervals as input and returns the root of the range tree.
\begin{verbatim}
BuildRangeIntervalTree(I)  
  iTree = ConstructIntervalTree(I)
  if size(I) == 1
    v = leaf(I[0], iTree)
  else
    (leftI, rightI, yMid) = SplitIntervalsOnMedianY(I)
    vLeft = BuildRangeIntervalTree(leftI)
    vRight = BuildRangeIntervalTree(rightI)
    v = node(yMid, vLeft, vRight, iTree)
  end
  return v
\end{verbatim}

\paragraph{Query}
We can adapt the \textsc{ 1DRangeQuery} algorithm from \cite{deBerg} by replacing the call to \textsc{ ReportSubtree} with a call to \textsc{ QueryIntervalTree}.


\subsection*{b. Correctness}
\paragraph{Construction}
TODO: Correctness of construction.

\paragraph{Query}
Correctness of the query follows directy from correctness of \textsc{ 1DRangeQuery} and \textsc{ QueryIntervalTree}.


\subsection*{c. Bounds}
\paragraph{Space}We have from lemma 5.6 in \cite{deBerg} that a 2D range tree uses $O(n \log n)$ storage. Our data structure is a 2d range tree where the associated 1D range tree has been replaced with an interval tree. Since both 1D range trees and interval trees needs $O(n)$ storage, lemma 5.6 also holds in this case, so we need $O(n \log n)$ storage.

\paragraph{Preprocessing}
We can build the primary structure in $O(n \log n)$. Each secondary structure can be build in $O(n \log n)$. This gives us the recurrence relation
\begin{align*}
  T(n) &= \twopartdef{O(1)}{n = 1}{2T(n/2) + O(n \log n)}{n > 1}
\end{align*}
From case 2 of the Master theorem we find that this has the solution
\begin{align*}
  T(n) = O(n \log^2 n)
\end{align*}
so we can build the range-interval tree in $O(n \log^2 n)$.

\paragraph{Query}
To query the tree we need to folow two paths from the root to the two leaves that bound the range from above and below. This takes $O(\log n)$ time and gives us $O(\log n)$ canonical subsets that needs to be queried. Each subset is stored in an interval tree which can be queried in $O(\log n)$ time. So we need $O(\log n)O(\log n) = O(\log^2 n)$ query time.




\chapter*{Assignment 10.6}
\paragraph{a} Instead of storing the intervals at the nodes, we simply store
the number of them instead, thus we only have to do a single binary search in
the segment tree and on the way down add the numbers of interval together. It
is clear that this take $O(log(n))$ time. The construction time is not
affected and is still $O(nlog(n))$.

\paragraph{b} The actual problem is that we want to do a range counting query
from $I_\textrm{mid}$ to $q$, based on the endpoint. We solved this problem in
last weeks assignment. So instead of arrays we attach range-counting trees.
Remember that each interval is hold only once by a node in the tree. Thus this
accumulated range counting queries takes at most $O(log(n))$ time. Thus the
running time and space consumption is not affected by replaceing the arrays by
range counting trees.

\chapter*{Assignment 10.9}
We need to find intervals on the real line that are completely contained in a given query interval. We can answer this query by constructing a 2D range tree. The primary structure is a balanced binary search tree indexed on the leftmost x-coordinate of the intervals. The associated structure is a list sorted on the rightmost x-coordinate of the intervals. When we need to report a subtree we just walk the list untill we find a rightmost x-coordinate that is not in the query interval.

\paragraph{Query time}
We use $O(\log n)$ time walking down the tree. For each node we visit we spend $O(m + 1)$ time, where $m$ is the number of intervals we report at the node. We only report an interval once, so in total we need $O(\log n + k)$ where $k$ is the number of reported nodes.

\paragraph{Storage}
The associated sorted list requires $O(n)$ storage. As in 10.1 we have a range tree where the associated structure has been replaced with one of the same size. So the total storage needed is $O(n \log n)$.


\chapter*{Assignment 10.10}

\printbibliography
\end{document}
